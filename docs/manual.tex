\documentclass{article}
\title{Debarcer: De-Barcoding and Error Correction}
\author{Paul Krzyzanowski\\
  Ontario Institute for Cancer Research\\
  Toronto, Ontario, Canada\\
  \texttt{paul.krzyzanowski@oicr.on.ca}}

% To regenerate the pdf, run pdflatex on a cluster node...

\overfullrule=2cm
\usepackage[letterpaper, margin=1in]{geometry}
\usepackage{listings}
\usepackage{verbatim}
\usepackage{tikz}
\usetikzlibrary{shapes,arrows}



\begin{document}
\maketitle


\section{Location}

	You can download the latest version of Debarcer from GitHub:

\begin{verbatim}
  https://github.com/oicr-gsi/debarcer/releases
\end{verbatim}

Once the archive has been extracted to a suitable directory, you
will need to copy the \\
\texttt{config\_files/debarcer.dotfile} 
to \texttt{\~\//.debarcer} as 
this configuration file is used throughout the pipeline. \\

The directory containing runDebarcer.sh must be in your PATH

\section{Using Debarcer}

\subsection{Requirements}

\begin{itemize}
  \item Ensure you have 16G memory available
  \item Note that a bam file for the current alignment will be generated in your work directory
  \item Currently, you need to have access to 'qsub' through SGE or OGS
\end{itemize}

\subsection{Running the package}

Once \texttt{runDebarcer.sh} is in your \texttt{\$PATH}, either manually or using modules, you can run it as follows:

\begin{verbatim}
  $ runDebarcer.sh -r -f [FASTQFILE] -n [SAMPLENAME] -o [OUTPUTDIR]
\end{verbatim}

Using modules is handy for running via qsub:

\begin{verbatim}
  qsub -N "Debarcer" -b y -cwd -l h_vmem=16g "module load debarcer/dev; \ 
	 runDebarcer.sh -r -f [FASTQFILE] -n [SAMPLENAME] -o [OUTPUTDIR]"
\end{verbatim}

There is a small test set of data included with the distribution:

\begin{verbatim}
  runDebarcer.sh -r -f [DebarcerRoot]/demodata/Sample_Test.R1.fastq.gz \
	-n Sample_Test -o ./testresults
\end{verbatim}

This test data was extracted from a sample that only contained the TP146 amplicon
\big(Specifically, \texttt{Sample 9297} from an internal dataset, to be posted somewhere in the future\big).
This file only contains reads which were from UID families of depth 20-30, inclusive.

\subsubsection{Timings}

\begin{itemize}
	\item The included test file should run in under 5 minutes.
	\item 2 million mapped reads: Under 15 minutes.
	\item 20 million mapped reads: ~200 minutes (~3h)
\end{itemize}


\subsection{Automating Analyses}

To help automate this whole process, the \\texttt{populateWorkDirectory.pl} script in the 
debarcer 'tools' directory can create a results hierarchy based on a
directory of fastq files and write the appropriate run scripts:

\begin{verbatim}
  mkdir -p ResultsRoot\fastqs
  cd ResultsRoot\fastqs
  find [location of the the fastq files] -name "*fastq.gz" -exec ln -s {} \;
  cd ..
  perl $DEBARCERHOME\tools\populateWorkDirectory.pl ./fastqs
\end{verbatim}


\begin{comment}
\section{Workflow}

% from http://www.texample.net/tikz/examples/simple-flow-chart/
% Define block styles
\tikzstyle{decision} = [diamond, draw, fill=blue!20, 
    text width=4.5em, text badly centered, node distance=3cm, inner sep=0pt]
\tikzstyle{block} = [rectangle, draw, fill=blue!20, 
    text width=5em, text centered, rounded corners, minimum height=4em]
\tikzstyle{line} = [draw, -latex']
\tikzstyle{cloud} = [draw, ellipse,fill=red!20, node distance=3cm,
    minimum height=2em]

\begin{center}    
\begin{tikzpicture}[node distance = 2cm, auto]
    % Place nodes
    \node [block] (align) {Perform bwa alignment};
    \node [cloud, left of=align] (fastq) {.fastq};
    \node [cloud, below of=align, node distance=3cm] (bam) {.bam};
    \node [block, below of=bam, node distance=3cm] (consensusY) {Generate consensus Y sequences from BAM};
    \node [block, left of=consensusY, node distance=3cm] (consensusX) {Generate consensus X sequences from BAM};
    \node [block, right of=consensusY, node distance=3cm] (consensusZ) {Generate consensus Z sequences from BAM};
    \node [block, below of=consensusY, node distance=3cm] (aggregate) {Aggregate Results};
    \node [block, below of=aggregate, node distance=3cm] (graphics) {Generate R graphics};
    % Draw edges
    \path [line] (fastq) -- (align);
    \path [line] (align) -- (bam);
    \path [line] (bam) -- (consensusX);
    \path [line] (bam) -- (consensusY);
    \path [line] (bam) -- (consensusZ);
    \path [line] (consensusX) -- (aggregate);
    \path [line] (consensusY) -- (aggregate);
    \path [line] (consensusZ) -- (aggregate);
    \path [line] (aggregate) -- (graphics);
    
\end{tikzpicture}
\end{center}
\end{comment}


\section{Output}

The output contains several different types of files.

Package results files with a tar command that wraps up the files found by:

\begin{verbatim}
	find . -name *.pdf -o -name "*bamPositionComposition*" -o \
	-name "*UID*" -o -name "*SummaryStatistics.txt" -o -name "*log"
\end{verbatim}

i.e.

\begin{verbatim}
	tar cvfz results_file_name.tar.gz `[find command goes in backticks]`
\end{verbatim}

This can also be done using the \texttt{packageResults.sh} file
in the Debarcer \texttt{tools} subdirectory. Simply execute that
script from the root of your results hierarchy.

\section{Miscellaneous}

Requirements:  Consensus calling script needs at least 16G memory.

\subsection{Revisions}

\begin{itemize}
  \item 0.3.0:  April 7, 2016. 
  \item 0.2.0:  July 27, 2015.  Svn Revision: 347.  
\end{itemize}

\end{document}
