\documentclass{article}
\title{Debarcer: De-Barcoding and Error Correction}
\author{Paul Krzyzanowski\\
  Ontario Institute for Cancer Research\\
  Toronto, Ontario, Canada\\
  \texttt{paul.krzyzanowski@oicr.on.ca}}

% To regenerate the pdf, run pdflatex on a cluster node...

\overfullrule=2cm
\usepackage[letterpaper, margin=1in]{geometry}

\usepackage{tikz}
\usetikzlibrary{shapes,arrows}



\begin{document}
\maketitle


\section{Location}

	You can export or checkout the whole package from the following location:

\begin{verbatim}
  $ svn export file:///u/pkrzyzanowski/svn/repo/projects \ 
	/EAC/molecular_barcoding/trunk/debarcer .
  $ svn co file:///u/pkrzyzanowski/svn/repo/projects \ 
	/EAC/molecular_barcoding/trunk/debarcer .
\end{verbatim}
  

\section{Using Debarcer}

\subsection{Requirements}

\begin{itemize}
  \item Ensure you have 16G memory available
  \item Note that a bam file for the current alignment will be generated in your work directory
  \item Currently, you need to have access to 'qsub' through SGE or OGS
\end{itemize}

\subsection{Running the package}

To generate a set of positional compositions, you can export 
Debarcer, set the BHOME environment variable, and run it like so:

\begin{verbatim}
  $ svn export file:///u/pkrzyzanowski/svn/repo/projects/EAC/ \ 
	molecular_barcoding/trunk/Debarcer .
  $ export BHOME $PWD
  $ runDebarcer.sh [FASTQFILE] [SAMPLENAME]
\end{verbatim}

Or if you're set up for using modules, you can load debarcer from anywhere and run it:

\begin{verbatim}
  $ module load debarcer/trunk
  $ runDebarcer.sh Sample_6.R1.fastq.gz Sample6
\end{verbatim}

Using modules is handy for running via qsub:

\begin{verbatim}
  qsub -N "BCS" -b y -cwd -l h_vmem=16g "module load debarcer/trunk; \ 
	runDebarcer.sh Sample_6.R1.fastq.gz Sample6"
\end{verbatim}

To run a test set of data, do:

\begin{verbatim}
  # ls /u/pkrzyzanowski/projects/simsenseq/debarcer/demodata # for more test files
  cp /u/pkrzyzanowski/projects/simsenseq/debarcer/demodata/Sample_9297.R1.fastq.gz .
  module load debarcer/trunk
  runDebarcer.sh Sample_9297.R1.fastq.gz Sample_9297
\end{verbatim}

Sample\_9297 is a 1-plex and Sample\_9292 is a 5-plex.


To help automate this whole process, there is a script in the 
debarcer 'tools' directory that will create a results hierarchy based on a
directory of fastq files and write the appropriate run scripts:

\begin{verbatim}
  mkdir -p ResultsRoot\fastqs
  cd ResultsRoot\fastqs
  find [location of the the fastq files] -name "*fastq.gz" -exec ln -s {} \;
  cd ..
  svn export file:///u/pkrzyzanowski/svn/repo/projects/EAC/ \ 
	molecular_barcoding/trunk/debarcer/tools/populateWorkDirectory.pl
  # Make any necessary changes to populateWorkDirectory.pl
  perl populateWorkDirectory.pl ./fastqs
  # Setup any debarcer.conf files in the results directories, if needed
  ./relaunchAllDebarcers.sh
\end{verbatim}



\section{Workflow}

% from http://www.texample.net/tikz/examples/simple-flow-chart/
% Define block styles
\tikzstyle{decision} = [diamond, draw, fill=blue!20, 
    text width=4.5em, text badly centered, node distance=3cm, inner sep=0pt]
\tikzstyle{block} = [rectangle, draw, fill=blue!20, 
    text width=5em, text centered, rounded corners, minimum height=4em]
\tikzstyle{line} = [draw, -latex']
\tikzstyle{cloud} = [draw, ellipse,fill=red!20, node distance=3cm,
    minimum height=2em]

\begin{center}    
\begin{tikzpicture}[node distance = 2cm, auto]
    % Place nodes
    \node [block] (align) {Perform bwa alignment};
    \node [cloud, left of=align] (fastq) {.fastq};
    \node [cloud, below of=align, node distance=3cm] (bam) {.bam};
    \node [block, below of=bam, node distance=3cm] (consensusY) {Generate consensus Y sequences from BAM};
    \node [block, left of=consensusY, node distance=3cm] (consensusX) {Generate consensus X sequences from BAM};
    \node [block, right of=consensusY, node distance=3cm] (consensusZ) {Generate consensus Z sequences from BAM};
    \node [block, below of=consensusY, node distance=3cm] (aggregate) {Aggregate Results};
    \node [block, below of=aggregate, node distance=3cm] (graphics) {Generate R graphics};
    % Draw edges
    \path [line] (fastq) -- (align);
    \path [line] (align) -- (bam);
    \path [line] (bam) -- (consensusX);
    \path [line] (bam) -- (consensusY);
    \path [line] (bam) -- (consensusZ);
    \path [line] (consensusX) -- (aggregate);
    \path [line] (consensusY) -- (aggregate);
    \path [line] (consensusZ) -- (aggregate);
    \path [line] (aggregate) -- (graphics);
    
\end{tikzpicture}
\end{center}

\section{Output}

The output contains several different types of files.

Package results files with a tar command that wraps up the files found by:

\begin{verbatim}
	find . -name *.pdf -o -name "*bamPositionComposition*" -o \
	-name "*UID*" -o -name "*SummaryStatistics.txt" -o -name "*log"
\end{verbatim}

i.e.

\begin{verbatim}
	tar cvfz results_file_name.tar.gz `[find command goes in backticks]`
\end{verbatim}


\section{Miscellaneous}

Requirements:  Consensus calling script needs at least 16G memory.

\subsection{Revisions}

0.2.0:  July 27, 2015.  Svn Revision: 347.  

\end{document}
